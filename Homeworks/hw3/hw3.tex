\documentclass{article}

\title{Homework 3}
\date{11-09-2023}
\author{Danny Zou}

\usepackage[left=.7in, right=.7in, top=0.5in, bottom=0.5in]{geometry}
\usepackage{amsmath}
\usepackage{indentfirst}
\usepackage{setspace}
\usepackage{graphicx}
\usepackage{multirow} 
\usepackage{amssymb}
\usepackage[table]{xcolor}

\newcommand{\boxedanswer}[1]{%

    \fbox{\large\textbf{#1}}%
}

\begin{document}
    \pagenumbering{gobble}
    \maketitle
    \onehalfspacing

    \section*{Boolean Algebra}

    \subsection*{1}

    \subsubsection*{(a) $A * (\overline{A} + B * B) + \overline{(B+A)} * (\overline{A} + B)$} 
    
    $A * (\overline{A} + B * B) + (\overline{B} * \overline{A}) * (\overline{A} + B)$ \hspace*{0.1in} DeMorgan's 

    $A * (\overline{A} + B) + (\overline{B} * \overline{A}) * (\overline{A} + B)$ \hspace*{0.1in} Idempotent Law ($B * B = B$)

    $(A + (\overline{B} * \overline{A}))(\overline{A} + B)$ \hspace*{0.1in} Distributive Law (Inversed)

    $(A + (\overline{A} * \overline{B}))(\overline{A} + B)$ \hspace*{0.1in} Commutative Law
    
    $(A + \overline{B})(\overline{A} + B)$ \hspace*{0.1in}  Redundacy Law ($A + (\overline{A} * \overline{B}) = A + \overline{B}$)

    $A(\overline{A} + B) + \overline{B}(\overline{A} + B)$ \hspace*{0.1in} Distributive Law

    $(A * \overline{A} + A * B) + \overline{B}(\overline{A} + B)$ \hspace*{0.1in} Distributive Law

    $(0 + A * B) + \overline{B}(\overline{A} + B)$ \hspace*{0.1in} Inverse Law ($A * \overline{A} = 0$)

    $(A * B) + \overline{B}(\overline{A} + B)$ \hspace*{0.1in} Identity Law ($0+A*B = A*B$)

    $(A * B) + (\overline{B}*\overline{A} + \overline{B} * B)$ \hspace*{0.1in} Distributive Law

    $(A * B) + (\overline{B}*\overline{A} + 0)$ \hspace*{0.1in} Inverse Law ($\overline{B} * B = 0$)

    $(A * B) + (\overline{B}*\overline{A})$ \hspace*{0.1in} Identity Law ($\overline{B}*\overline{A} + 0 = \overline{B}*\overline{A}$)

    \vspace*{0.1in}

    \boxedanswer{$A * B + \overline{B}*\overline{A}$}

    \subsubsection*{(b) $\overline{C * B} + (A * B * C) + \overline{A + C + \overline{B}}$}

    $\overline{C} + \overline{B} + (A * B * C) + \overline{A + C + \overline{B}}$ \hspace*{0.1in} DeMorgan's

    $\overline{C} + (A * B * C) + \overline{B} + \overline{A + C + \overline{B}}$ \hspace*{0.1in} Commutative Law

    $\overline{C} + (C * A * B) + \overline{B} + \overline{A + C + \overline{B}}$ \hspace*{0.1in} Commutative Law

    $\overline{C} + ( A * B) + \overline{B} + \overline{A + C + \overline{B}}$ \hspace*{0.1in} Redundacy Law ($\overline{C} + (C * A * B) = \overline{C} + (A * B))$

    $\overline{C} + \overline{B} + ( A * B) + \overline{A + C + \overline{B}}$ \hspace*{0.1in} Commutative Law

    $\overline{C} + \overline{B} + (B * A) + \overline{A + C + \overline{B}}$ \hspace*{0.1in} Commutative Law
    
    $\overline{C} + (\overline{B} + A) + \overline{A + C + \overline{B}}$ \hspace*{0.1in} Redundacy Law ($\overline{B} + (B*A) = \overline{B} + A$)

    $\overline{C} + (\overline{B} + A) + \overline{A} * \overline{C} * \overline{\overline{B}}$ \hspace*{0.1in} DeMorgan's

    $\overline{C} + (\overline{B} + A) + \overline{A} * \overline{C} * B$ \hspace*{0.1in} Double Negation Law

    $\overline{C} + (\overline{A} * \overline{C} * B)+  (\overline{B} + A)  $ \hspace*{0.1in} Commutative Law

    $\overline{C} + (\overline{C} * \overline{A} * B)+  (\overline{B} + A)  $ \hspace*{0.1in} Commutative Law

    $\overline{C} +  (\overline{B} + A)  $ \hspace*{0.1in} Absorption Law ($\overline{C} + (\overline{C} * \overline{A} * B) = \overline{C}$)

    \vspace*{0.1in}

    \boxedanswer{$\overline{C} + \overline{B} + A$}

    \newpage

    \subsubsection*{(c) $(A+B) * (\overline{A} + C) * (\overline{C} + B)$}

    $(A(\overline{A} + C) + B(\overline{A} + C))* (\overline{C} + B)$ \hspace*{0.1in} Distributive Law

    $((A*\overline{A} + A*C)+B(\overline{A} + C))* (\overline{C} + B)$ \hspace*{0.1in} Distributive Law

    $(A*\overline{A} + A*C+ B*\overline{A} + B*C)* (\overline{C} + B)$ \hspace*{0.1in} Distributive Law

    $((A*\overline{A}*\overline{C}) + (A*C*\overline{C})+ (B*\overline{A}*\overline{C}) + (B*C*\overline{C})) + $
    
    $((A*\overline{A}*B) + (A*C*B)+ (B*\overline{A}*B) + (B*C*B))$ \hspace*{0.1in} Distributive Law

    $((A*\overline{A}*\overline{C}) + (A*C*\overline{C})+ (B*\overline{A}*\overline{C}) + (B*C*\overline{C})) + $
    
    $((A*\overline{A}*B) + (A*C*B)+ (B*\overline{A}*B) + (B*B*C))$ \hspace*{0.1in} Commutative Law

    $((A*\overline{A}*\overline{C}) + (A*C*\overline{C})+ (B*\overline{A}*\overline{C}) + (B*C*\overline{C})) + $
    
    $((A*\overline{A}*B) + (A*C*B)+ (B*\overline{A}*B) + (B*C))$ \hspace*{0.1in} Idempotent Law

    $((A*\overline{A}*\overline{C}) + (A*C*\overline{C})+ (B*\overline{A}*\overline{C}) + (B*C*\overline{C})) + $
    
    $((A*\overline{A}*B) + (A*C*B)+ (B*B*\overline{A}) + (B*C))$ \hspace*{0.1in} Commutative Law

    $((A*\overline{A}*\overline{C}) + (A*C*\overline{C})+ (B*\overline{A}*\overline{C}) + (B*C*\overline{C})) + $
    
    $((A*\overline{A}*B) + (A*C*B)+ (B*\overline{A}) + (B*C))$ \hspace*{0.1in} Idempotent Law

    $((0*\overline{C}) + (A*C*\overline{C})+ (B*\overline{A}*\overline{C}) + (B*C*\overline{C})) + $
    
    $((A*\overline{A}*B) + (A*C*B)+ (B*\overline{A}) + (B*C))$ \hspace*{0.1in} Inverse Law

    $((0*\overline{C}) + (A*0)+ (B*\overline{A}*\overline{C}) + (B*C*\overline{C})) + $
    
    $((A*\overline{A}*B) + (A*C*B)+ (B*\overline{A}) + (B*C))$ \hspace*{0.1in} Inverse Law

    $((0*\overline{C}) + (A*0)+ (B*\overline{A}*\overline{C}) + (B*0)) + $
    
    $((A*\overline{A}*B) + (A*C*B)+ (B*\overline{A}) + (B*C))$ \hspace*{0.1in} Inverse Law

    $((0*\overline{C}) + (A*0)+ (B*\overline{A}*\overline{C}) + (B*0)) + $
    
    $((0*B) + (A*C*B)+ (B*\overline{A}) + (B*C))$ \hspace*{0.1in} Inverse Law

    $((0) + (0)+ (B*\overline{A}*\overline{C}) + (0)) + $
    
    $((0) + (A*C*B)+ (B*\overline{A}) + (B*C))$ \hspace*{0.1in} Law of Zeros (Too much to do them one by one ...)

    $(B*\overline{A}*\overline{C}) + (A*C*B)+ (B*\overline{A}) + (B*C)$ \hspace*{0.1in} Identity Law (Too much to do them one by one...)

    $(B*\overline{A})+ (B*\overline{A}*\overline{C}) + (A*C*B) + (B*C)$ \hspace*{0.1in} Commutative Law

    $(B*\overline{A}) + (A*C*B) + (B*C)$ \hspace*{0.1in} Absorption Law (($B*\overline{A})+ (B*\overline{A}*\overline{C}) = (B*\overline{A})$)

    $(B*\overline{A}) + (B*C) + (A*C*B)$ \hspace*{0.1in} Commutative Law

    $(B*\overline{A}) + (B*C) + (B*C*A)$ \hspace*{0.1in} Commutative Law

    $(B*\overline{A}) + (B*C)$ \hspace*{0.1in} Absorption Law ($(B*C) + (B*C*A) = (B*C)$)

    $B*(\overline{A}+C)$ Distributive Law (Inversed)

    \vspace*{0.1in}

    \boxedanswer{$B*(\overline{A}+C)$}

    \newpage

    \subsection*{2}

    \subsubsection*{(a) $(\overline{A} + C) * (\overline{B} + D + A) * (D + A * \overline{C}) * (\overline{D} + A) = 1$}

    $(\overline{A} + C) * (\overline{B} + D + A) * (D + \overline{\overline{A} + \overline{\overline{C}}}) * (\overline{D} + A)$ \hspace*{0.1in} DeMorgan's

    $(\overline{A} + C) * (\overline{B} + D + A) * (D + \overline{\overline{A} + C}) * (\overline{D} + A)$ \hspace*{0.1in} Double Negation Law

    $(\overline{A} + C) * (D + \overline{\overline{A} + C}) * (\overline{B} + D + A) * (\overline{D} + A)$ \hspace*{0.1in} Commutative Law

    $(\overline{A} + C) * (\overline{\overline{A} + C} + D) * (\overline{B} + D + A) * (\overline{D} + A)$ \hspace*{0.1in} Commutative Law

    $((\overline{A} + C) * D) * (\overline{B} + D + A) * (\overline{D} + A)$ \hspace*{0.1in} Redundacy Law ($(\overline{A} + C) * (\overline{\overline{A} + C} + D) = (\overline{A} + C) * D $)

    $(\overline{A} + C) * D * (D + A + \overline{B}) * (\overline{D} + A)$ \hspace*{0.1in} Commutative Law

    $(\overline{A} + C) * D * (\overline{D} + A)$ \hspace*{0.1in} Absorption Law ($D * (D + A + \overline{B}) = D$)

    $(\overline{A} + C) * (D * A)$ \hspace*{0.1in} Redundacy Law ($D * (\overline{D} + A) = D * A$)

    $A * (\overline{A} + C) * D$ \hspace*{0.1in} Commutative Law

    $A * C * D$ \hspace*{0.1in} Redundacy Law ($A * (\overline{A} + C) = A * C$)

    $A * C * D = 1$

    \vspace*{0.1in}

    \boxedanswer{A = 1, C = 1, D = 1, B = 0 or 1}

    \subsubsection*{(b) $(((\overline{K} * L * N) * (L+M)) + ((\overline{K}+L+N) * (K*\overline{L}*\overline{M}))) * (\overline{K} + \overline{N}) = 1$}

    $((\overline{K} * L * (L+M) * N) + ((\overline{K}+L+N) * (K*\overline{L}*\overline{M}))) * (\overline{K} + \overline{N})$ \hspace*{0.1in} Commutative Law

    $((\overline{K} * L * N) + ((\overline{K}+L+N) * (K*\overline{L}*\overline{M}))) * (\overline{K} + \overline{N})$ \hspace*{0.1in} Absorption Law ($L * (L+M) = L$)

    $((\overline{K} * L * N) + (K * (\overline{K}+L+N) * \overline{L}*\overline{M})) * (\overline{K} + \overline{N})$ \hspace*{0.1in} Commutative Law

    $((\overline{K} * L * N) + (K * (L+N) * \overline{L}*\overline{M})) * (\overline{K} + \overline{N})$ \hspace*{0.1in} Redundacy Law ($K * (\overline{K}+L+N) = K*(L+N)$)

    $((\overline{K} * L * N) + (K * \overline{L} * (L+N)*\overline{M})) * (\overline{K} + \overline{N})$ \hspace*{0.1in} Commutative Law

    $((\overline{K} * L * N) + (K * \overline{L}* N *\overline{M})) * (\overline{K} + \overline{N})$ \hspace*{0.1in} Redundacy Law ($\overline{L} * (L+N) = L * N$) 

    $N * ((\overline{K} * L) + (K * \overline{L} *\overline{M})) * (\overline{K} + \overline{N})$ \hspace*{0.1in} Distributive Law (Inversed)

    $N * (\overline{K} + \overline{N}) * ((\overline{K} * L) + (K * \overline{L} *\overline{M}))$ \hspace*{0.1in} Commutative Law

    $N * (\overline{N} + \overline{K}) * ((\overline{K} * L) + (K * \overline{L} *\overline{M}))$ \hspace*{0.1in} Commutative Law

    $N * \overline{K} * ((\overline{K} * L) + (K * \overline{L} *\overline{M}))$ \hspace*{0.1in} Redundacy Law ($N * (\overline{N} + \overline{K}) = N * \overline{K}$)

    $(\overline{K} * L * N * \overline{K} ) + (K * \overline{L} *\overline{M} * N * \overline{K} )$ \hspace*{0.1in} Distributive Law

    $(\overline{K} * \overline{K} * L * N ) + (K * \overline{L} *\overline{M} * N * \overline{K} )$ \hspace*{0.1in} Commutative Law

    $(\overline{K} * L * N ) + (K * \overline{L} *\overline{M} * N * \overline{K} )$ \hspace*{0.1in} Idempotent Law

    $(\overline{K} * L * N ) + (K * \overline{K}  * \overline{L} *\overline{M} * N)$ \hspace*{0.1in} Commutative Law

    $(\overline{K} * L * N ) + (0 * \overline{L} *\overline{M} * N)$ \hspace*{0.1in} Inverse Law

    $(\overline{K} * L * N ) + (0)$ \hspace*{0.1in} Law of Zeros

    $(\overline{K} * L * N )$ \hspace*{0.1in} Identity Law

    $\overline{K} * L * N = 1$ 

    \boxedanswer{K = 0, L = 1, N = 1, M = 0 or 1}

    \newpage

    \subsection*{3}

    \subsubsection*{Truth Table}

    \begin{flushleft}
    \begin{tabular}{|c|c|c|c|}
    \hline
    $X$ & $Y$ & $Z$ & $Q$ \\
    \hline
    0 & 0 & 0 & 0\\
    0 & 0 & 1 & 1\\
    0 & 1 & 0 & 1\\
    0 & 1 & 1 & 0\\
    1 & 0 & 0 & 1\\
    1 & 0 & 1 & 0\\
    1 & 1 & 0 & 1\\
    1 & 1 & 1 & 0\\
    \hline
    \end{tabular}
    \end{flushleft}

    \subsubsection*{K-Map}
    
    \begin{table}[h]
        
        \begin{tabular}{|c|c|c|c|c|}
            \hline
            \multirow{2}{*}{\textbf{Z}} & \multicolumn{4}{c|}{\textbf{XY}} \\
            \cline{2-5}
             & \textbf{00} & \textbf{01} & \textbf{11} & \textbf{10} \\
            \hline
            0 & 0& 1& 1& 1\\
            \hline
            1 & 1& 0& 0& 0\\
            \hline
        \end{tabular}
    \end{table}

    \begin{table}[h]
        \begin{tabular}{|c|c|c|c|c|}
            \hline
            \multirow{2}{*}{\textbf{Z}} & \multicolumn{4}{c|}{\textbf{XY}} \\
            \cline{2-5}
             & \textbf{00} & \textbf{01} & \textbf{11} & \textbf{10} \\
            \hline
            0 & 0& \cellcolor{yellow!}1& \cellcolor{yellow}1& 1\\

            \hline
            1 & 1& 0& 0& 0\\
            \hline
        \end{tabular}
    \end{table}
    \noindent
    The merge shown by the yellow means "if Z is 0 and Y is 1, it doesn't matter what the value of X is."

    \noindent
    In the function, it expresses -- $X * Y *\overline{Z} + \overline{X} * Y * \overline{Z} = Y * \overline{Z}$

    \begin{table}[h]
        \begin{tabular}{|c|c|c|c|c|}
            \hline
            \multirow{2}{*}{\textbf{Z}} & \multicolumn{4}{c|}{\textbf{XY}} \\
            \cline{2-5}
             & \textbf{00} & \textbf{01} & \textbf{11} & \textbf{10} \\
            \hline
            0 & 0& 1& \cellcolor{green}1& \cellcolor{green!}1\\

            \hline
            1 & 1& 0& 0& 0\\
            \hline
        \end{tabular}
    \end{table}
    \noindent
    The merge shown by the yellow means "if Z is 0 and X is 1, it doesn't matter what the value of Y is."

    \noindent
    In the function, it expresses -- $X*Y*\overline{Z} + X*\overline{Y}*\overline{Z} = X * \overline{Z}$

    \vspace*{0.1in}

    \noindent
    After applying those two reductions, we get

    \vspace*{0.1in}

    Original : $\overline{X} * \overline{Y} * Z + X * Y * \overline{Z} + \overline{X} * Y * \overline{Z} + X * \overline{Y} * \overline{Z}$

    \vspace*{0.1in}

    Reduced:
    \vspace*{0.1in}
    \boxedanswer{$Y*\overline{Z} + X * \overline{Z} + \overline{X}*\overline{Y} * Z$}

    \newpage

    \section*{Logical Circuits}

    \subsection*{4}

    To obtain the sums of products representation, we look for the rows where the output is 1 and then form a product 
    
    term for each row.

    From the truth table, we get 


    \vspace*{0.1in}

    \boxedanswer{$(\overline{A} * \overline{B} * \overline{C}) + (\overline{A} * B * \overline{C}) + (\overline{A} * B * C) + (A*B*C)$}

    \subsection*{6}

    \includegraphics*[width=0.45\textwidth]{circuit1.png}

    \small When input A = 0, B = 0, C = 0, the output is 1, like in the truth table (Also satisfies expression ($\overline{A} * \overline{B} * \overline{C}$) as it should)

    \vspace*{0.1in}

    \includegraphics*[width=0.45\textwidth]{circuit2.png}

    \small When input A = 0, B = 0, C = 1, the output is 0, like in the truth table (Doesn't satisfy none of the expressions)

    \vspace*{0.1in}

    \includegraphics*[width=0.45\textwidth]{circuit3.png}

    \small When input A = 0, B = 1, C = 0, the output is 1, like in the truth table (Also satisfies expression ($\overline{A} * B * \overline{C}$))

    \includegraphics*[width=0.45\textwidth]{circuit4.png}

    \small When input A = 0, B = 1, C = 1, the output is 1, like in the truth table (Also satisfies expression ($\overline{A} * B * C$))

    \vspace*{0.1in}

    \includegraphics*[width=0.45\textwidth]{circuit5.png}

    \small When input A = 1, B = 0, C = 0, the output is 0, like in the truth table (Doesn't satisfy none of the expressions)

    \vspace*{0.1in}

    \includegraphics*[width=0.45\textwidth]{circuit6.png}

    \small When input A = 1, B = 0, C = 1, the output is 0, like in the truth table (Doesn't satisfy none of the expressions)
    
    \newpage

    \vspace*{0.1in}

    \includegraphics*[width=0.45\textwidth]{circuit7.png}

    \small When input A = 1, B = 1, C = 0, the output is 0, like in the truth table (Doesn't satisfy none of the expressions)

    \vspace*{0.1in}

    \includegraphics*[width=0.45\textwidth]{circuit8.png}

    \small When input A = 1, B = 1, C = 1, the output is 1, like in the truth table (Also satisfies expression ($A*B*C$))

    \vspace*{0.2in}

    After going through all possible combination of inputs, I find that the expected output with each different set of inputs is exactly like it is in the truth table. When the output is 1, the appropriate AND components also lights up, matching the appropriate expression in the boolean function (i.e., ($A*B*C$) is an expression that lights up when A = 1, B = 1, C = 1). Through these findings, I believe there is sufficient proof that my logical circuit does in fact implement the required Boolean function, and that the boolean function also correctly represents the truth table.

    \subsection*{7}

    First of all

    \vspace*{0.1in}

    \includegraphics*[width=0.5\textwidth]{NOTA.png}

    And

    \vspace*{0.1in}

    \includegraphics*[width=0.5\textwidth]{A+0.png}

    Using these logics, we can implement \textbf{AND\{(A*B)\}} using only NOR Gates by

    \vspace*{0.1in}

    \includegraphics*[width=0.5\textwidth]{AND.png}

    We can also implement \textbf{OR\{(A+B)\}} using only NOR Gates by

    \vspace*{0.1in}

    \includegraphics*[width=0.5\textwidth]{or.png}

    And finally implement \textbf{NOT\{$\overline{A}$\}} simply as

    \vspace*{0.1in}

    \includegraphics*[width=0.5\textwidth]{not.png}

    Thus we have shown \textbf{NOR\{$\overline{(A+B)}$\}} is functionally complete by giving the logical circuit equivalents to each AND(A*B), OR(A+B) and NOT($\overline{A}$).

    \newpage

    \section*{Numerical Conversions and Arithmetic}

    \subsection*{8}

    \subsubsection*{50.4375}

    $\underline{50}.4375$

    $50 = \underline{25} * 2 + \mathbf{0}$

    $25 = \underline{12} * 2 + \mathbf{1}$

    $12 = \underline{6} * 2 + \mathbf{0}$

    $6 = \underline{3} * 2 + \mathbf{0}$

    $3 = \underline{1} * 2 + \mathbf{1}$

    $1 = 0 * 2 + \mathbf{1}$

    We get $\mathbf{110010}$, which is 50 in binary representation

    \vspace*{0.1in}

    $50\underline{.4375}$

    $.4375 * 2 = \mathbf{0}.875$

    $.0875 * 2 = \mathbf{1}.75$

    $.75 * 2 = \mathbf{1}.5$

    $.5 * 2 = \mathbf{1}.0$

    We get $\mathbf{0111}$, which is .4375 in binary representation

    \vspace*{0.1in}

    Combining, we get -- $\mathbf{110010.0111}$

    Convert to scientific notation, we get -- $\mathbf{1\underline{.100100111} * 2^{\underline{5}}}$

    Where $\mathbf{100100111}$ is the mantisa

    And $\mathbf{5}$ is the exponent

    Convert the exponent to the bias(127 + 5 = $\mathbf{132}$) in binary, we get $\mathbf{1000\: 0100}$

    We have everything we need, lets list them out

    \vspace*{0.1in}

    Sign = $\mathbf{0}$, because positive

    Exponent Bias Binary = $\mathbf{1000\: 0100}$

    Mantisa/Fraction = $\mathbf{10010011100000000000000}$

    Combine to get Single Precision IEEE 754 Floating Point Representation 
    
    Where Sign = 1 bit, Exponent = 8 bits, Mantisa/Fraction = 23 bits

    \vspace*{0.1in}

    \boxedanswer{0100 0010 0100 1001 1100 0000 0000 0000}

    Get the hexidecimal value by converting each set of 4 binaries to respective value

    \vspace*{0.1in}

    \boxedanswer{32-bit Hexadecimal -- 0x4249c000}

    \subsubsection*{0.0}

    Sign = $\mathbf{0}$, because positive

    Exponent = 0, Bias Binary = $\mathbf{0000\: 0000}$

    Mantisa/Fraction = $\mathbf{00000000000000000000000}$

    Combine to get Single Precision IEEE 754 Floating Point Representation 
    
    Where Sign = 1 bit, Exponent = 8 bits, Mantisa/Fraction = 23 bits

    \vspace*{0.1in}

    \boxedanswer{0000 0000 0000 0000 0000 0000 0000 0000}
    
    Get the hexidecimal value by converting each set of 4 binaries to respective value

    \vspace*{0.1in}

    \boxedanswer{32-bit Hexadecimal -- 0x00000000}

    \newpage

    \subsubsection*{-Infinity}

    Sign = $\mathbf{1}$, because Negative

    Exponent = 128, Bias(128 + 127 = $\mathbf{255}$) Binary = $\mathbf{1111\: 1111}$

    Mantisa/Fraction = $\mathbf{00000000000000000000000}$

    Combine to get Single Precision IEEE 754 Floating Point Representation 
    
    Where Sign = 1 bit, Exponent = 8 bits, Mantisa/Fraction = 23 bits

    \vspace*{0.1in}

    \boxedanswer{1111 1111 1000 0000 0000 0000 0000 0000}
    
    Get the hexidecimal value by converting each set of 4 binaries to respective value

    \vspace*{0.1in}

    \boxedanswer{32-bit Hexadecimal -- 1xff800000}

    \subsubsection*{1.0000001}

    $\underline{1}.0000001$

    $1 = 0 * 2 + \mathbf{1}$

    We get $\mathbf{1}$, which is 1 in binary representation

    \vspace*{0.1in}

    $1\underline{.0000001}$

    $.0000001 * 2 = \mathbf{0}.0000002$

    $.0000002 * 2 = \mathbf{0}.0000004$

    $.0000004 * 2 = \mathbf{0}.0000008$

    $.0000008 * 2 = \mathbf{0}.0000016$

    $.0000016 * 2 = \mathbf{0}.0000032$

    $.0000032 * 2 = \mathbf{0}.0000064$

    $.0000064 * 2 = \mathbf{0}.0000128$

    $.0000128 * 2 = \mathbf{0}.0000256$

    $.0000256 * 2 = \mathbf{0}.0000512$

    $.0000512 * 2 = \mathbf{0}.0001024$

    $.0001024 * 2 = \mathbf{0}.0002048$

    $.0002048 * 2 = \mathbf{0}.0004096$

    $.0004096 * 2 = \mathbf{0}.0008192$

    $.0008192 * 2 = \mathbf{0}.0016384$

    $.0016384 * 2 = \mathbf{0}.0032768$

    $.0032768 * 2 = \mathbf{0}.0065536$

    $.0065536 * 2 = \mathbf{0}.0131072$

    $.0131072 * 2 = \mathbf{0}.0262144$

    $.0262144 * 2 = \mathbf{0}.0524288$

    $.0524288 * 2 = \mathbf{0}.1048576$

    $.1048576 * 2 = \mathbf{0}.2097152$

    $.2097152 * 2 = \mathbf{0}.4194304$

    $.4194304 * 2 = \mathbf{0}.8388608$

    $.8388608 * 2 = \mathbf{1}.6777216$
    
    We get $\mathbf{\overline{00000000000000000000001}}$, which is 0.0000001 in binary representation
    
    \vspace*{0.1in}

    Combining, we get -- $\mathbf{1.\overline{00000000000000000000001}}$

    Convert to scientific notation, we get -- $\mathbf{1.\underline{00000000000000000000001} * 2^{\underline{0}}}$

    Where $\mathbf{00000000000000000000001}$ is the mantisa/fraction

    And $\mathbf{0}$ is the exponent

    Convert the exponent to the bias(127 + 0 = $\mathbf{127}$) in binary, we get $\mathbf{0111\:1111}$

    We have everything we need, lets list them out

    \vspace*{0.1in}

    Sign = $\mathbf{0}$, because positive

    Exponent Bias Binary = $\mathbf{0111\:1111}$

    Mantisa/Fraction = $\mathbf{00000000000000000000001}$

    Combine to get Single Precision IEEE 754 Floating Point representation

    Where Sign = 1 bit, Exponent = 8 bits, Mantisa/Fraction = 23 bits

    \vspace*{0.1in}

    \boxedanswer{0011 1111 1000 0000 0000 0000 0000 0001}

    Get the hexidecimal value by converting each set of 4 binaries to respective value

    \vspace*{0.1in}

    \boxedanswer{32-bit Hexadecimal -- 0x3f800001}

    \vspace*{0.2in}

    \subsection*{9}

    \subsubsection*{0xc349a000}

    IEEE 7354 Representation -- $\mathbf{1100\: 0011\: 0100\: 1001\: 1010\: 0000\: 0000\: 0000}$

    Sign = 1

    Exponent Bias Binary = $\mathbf{1000\: 0110}$

    Mantisa/Fraction = $\mathbf{10010011010000000000000}$

    Exponent Bias Decimal Value = 134

    Exponent = 134 - 127 = $\mathbf{7}$

    Fraction Value From Mantisa = $2^{-1} + 2^{-4} + 2^{-7} + 2^{-8}+ 2^{-10}$ = $\frac{589}{1024}$

    \vspace*{0.1in}

    Equation to plug into \hspace*{0.1in} $-1^{Sign} * (1 + (Fraction Value)) * 2^{Exponent}$

    Plug in, we get

    $-1^1 * (1 + \frac{589}{1024}) * 2^7$ = $-(\frac{1613}{1024}) * 2^7 = -201.625$

    \vspace*{0.1in}

    \boxedanswer{-201.625}

    \subsubsection*{0xffe00001}

    IEEE 7354 Representation -- $\mathbf{1111\: 1111\: 1110\: 0000\: 0000\: 0000\: 0000\: 0001}$

    Sign = 1

    Exponent Bias Binary = $\mathbf{1111\: 1111}$

    Mantisa/Fraction = $\mathbf{110 0000 0000 0000 0000 0001}$

    Because the exponent field is all 1s and the fraction field is non-zero, we can conclude what we are trying to find is \textbf{NaN}

    \vspace*{0.1in}

    \boxedanswer{NaN(Not a Number)}

    \subsubsection*{0x80000000}

    IEEE 7354 Representation -- $\mathbf{1000\:0000\:0000\:0000\:0000\:0000\:0000\:0000}$

    Sign = 1

    Exponent Bias Binary = $\mathbf{0000\:0000}$

    Mantisa/Fraction = $\mathbf{00000000000000000000000}$

    Both Fraction Value and Exponent Value is 0, so Decimal Value = -0, negative because sign is 1

    \vspace*{0.1in}

    \boxedanswer{-0}

    \newpage

    \subsubsection*{0x00400000}

    IEEE 7354 Representation -- $\mathbf{0000\: 0000\: 0100\: 0000\:0000\:0000\:0000\:0000}$

    Sign = 0

    Exponent Bias Binary = $\mathbf{0000\:0000}$

    Mantisa/Fraction = $\mathbf{10000000000000000000000}$

    Exponent Bias Decimal Value = 0

    Fraction Value From Mantisa = $2^{-1} = \frac{1}{2}$

    Because Fraction Value is non-zero and the Exponent Value is 0, the number is \textbf{subnormal}

    We can then use this formula \hspace*{0.2in} $-1^{Sign} * (Fraction Value) * 2^{-126}$

    Plug in, we get

    $-1^0 * (\frac{1}{2}) * 2^{-126} = \frac{1}{2} * 2^{-126} = \frac{1}{2^{127}}$ \hspace*{0.3in} \boxed{\frac{1}{2^{127}}}

    \subsection*{10}

    The use of 2's complement representation for negative numbers in computer arithmetic is primarily due to its simplicity and efficiency in performing arithmetic operations, especially addition and subtraction. One such reason is.
    
    Using 2's complement simplifies arithmetic operations because addition and subtraction can be performed using the same hardware for both positive and negative numbers without the need for separate logic for subtraction.

    Another reason is that 2's complement naturally supports modular arithmetic, making it easier to perform operations like addition, subtraction, and multiplication within a finite range.

    \vspace*{0.2in}

    For example: Let's add 3 and -5

    3 in binary is \textbf{0000 0011}

    -5 in binary is \textbf{1111 1011}

    \: \: 0000 0011

    + 1111 1011

    \: 1111 1110 

    This showcases how addition between positive and negative numbers is performed using the same binary arithmetic, making the process simpler and more efficient for computers.
    



\end{document}